%%%%%%%%%%%%%%%%%%%%%%%%%%%%%%%%%%%%%%%%%
% Medium Length Professional CV
% LaTeX Template
% Version 2.0 (8/5/13)
%
% This template has been downloaded from:
% http://www.LaTeXTemplates.com
%
% Original author:
% Trey Hunner (http://www.treyhunner.com/)
%
% Important note:
% This template requires the resume.cls file to be in the same directory as the
% .tex file. The resume.cls file provides the resume style used for structuring the
% document.
%
%%%%%%%%%%%%%%%%%%%%%%%%%%%%%%%%%%%%%%%%%

%----------------------------------------------------------------------------------------
%	PACKAGES AND OTHER DOCUMENT CONFIGURATIONS
%----------------------------------------------------------------------------------------

\documentclass{resume} % Use the custom resume.cls style

\usepackage[left=0.75in,top=0.5in,right=0.75in,bottom=0.75in]{geometry} % Document margins
\usepackage{hyperref}
\usepackage{xcolor}
\definecolor{darkred}{RGB}{139 0 0}
\renewcommand{\familydefault}{\sfdefault}

\hypersetup{
    colorlinks = true,
    urlcolor = darkred
}

\name{Pranav Ravichandran} % Your name
\address{\textit{Full-stack Engineering} | \textit{Continuous Delivery} | \textit{Build Infrastructure} | \textit{Web Scalability}}
\address{Sunnyvale, CA \\ {\href{mailto:me@onloop.net}{me@onloop.net}} \\ {\href{http://onloop.net}{http://onloop.net}} \\ {\href{https://github.com/pranavrc}{https://github.com/pranavrc}}} % Your phone number and email

\begin{document}

%----------------------------------------------------------------------------------------
%	EDUCATION SECTION
%----------------------------------------------------------------------------------------

\begin{rSection}{Education}
\vspace{2mm}

\begin{rSubsection}{University of Florida}{Dec 2016}{Master of Science in Computer Science. GPA: 3.51/4.00
}{Gainesville, FL}
\item {\bf {\href{https://github.com/pranavrc/fitness/}{Fitness}}} A genetic programming system using Common Lisp.
\item {\bf {\href{https://github.com/pranavrc/clause/}{Clause}}} A Common Lisp resolution refutation framework.
\item {\bf {\href{https://github.com/pranavrc/tagspace/}{Tagspace}}} Web query classification using inter-article relationships in Wikipedia.
\end{rSubsection}

\begin{rSubsection}{Anna University}{2009-2013}{Bachelor of Engineering in Electrical and Electronics Engineering. GPA: 7.44/10.00}{Chennai, India}
\item {\bf {\href{https://onloop.net/hairyplotter/}{Thesis}}} Gestural interface with Electrooculogram-based eye gesture detection and classification.
\end{rSubsection}

\end{rSection}

%----------------------------------------------------------------------------------------
%	WORK EXPERIENCE SECTION
%----------------------------------------------------------------------------------------

\begin{rSection}{Experience}
\vspace{2mm}

\begin{rSubsection}{\href{https://yahoo.com/}{Yahoo}}{Feb 2017 - Current}{Software Development Engineer, Developer Platforms and Services}{Sunnyvale, CA}
\item Developed new features, build systems and infrastructure for Yahoo's centralized continuous delivery platform.
\item Served users on call and offered solutions for technical issues as a Developer Operations engineer.
\item Presented a conference talk on iOS mobile build infrastructure at Yahoo's annual Tech Pulse 2017 event.
\end{rSubsection}

%------------------------------------------------

\begin{rSubsection}{\href{https://yahoo.com/}{Yahoo}}{May - Aug 2016}{Software Engineering Intern, Developer Platforms and Services}{Sunnyvale, CA}
\item Implemented new features and services for the Platform As A Service team.
\item Worked with the mobile build infrastructure team to design and implement the next generation of iOS build systems and tools.
\end{rSubsection}

%------------------------------------------------

\begin{rSubsection}{\href{https://happyfox.com/}{HappyFox Inc.}}{Jan 2014 - May 2015}{Software Developer (Full stack)}{Chennai, India}
\item Implemented features, enhancements and integrations for HappyFox, HappyFox Chat and related components.
\item Created a decoupled REST-based task-server with Celery and RabbitMQ, to handle deferred services.
\item Introduced contextual search for knowledge base articles using Latent Semantic Analysis.
\end{rSubsection}

%------------------------------------------------

\begin{rSubsection}{\href{https://www.google-melange.com/gsoc/project/details/google/gsoc2011/pranavrc/5757334940811264}{Google Summer of Code, Google}}{Apr - Sep 2011}{Student Developer}{Remote}
\item Integrated SMARTS, a game AI library, into KDE Gluon, giving it the capabilities of a Behavior Tree system.
\item Developed a KPart GUI to plug the SMARTS Behavior Tree editor into Gluon.
\end{rSubsection}

\end{rSection}

%-------------------------------------------------
% OPEN SOURCE CONTRIBUTIONS
%-------------------------------------------------

\begin{rSection}{Open Source Work}
\vspace{2mm}

\begin{rSubsection}{\href{https://bugzilla.mozilla.org/user\%5Fprofile?user\%5Fid=431664}{Mozilla Firefox}}{2012-2013}{}{}
Developed a code commenting and uncommenting feature for Mozilla Firefox developer tools, implemented W3C spec restrictions on WebIDL dictionary member types, and contributed other patches and fixes.
\end{rSubsection}

\begin{rSubsection}{\href{https://github.com/mdoege/PySynth/commits?author=pranavrc}{PySynth}}{July 2012}{}{}
Authored an interactive Command Line Interpreter to parse Musical Notation and render tunes on the fly, implemented multi-library support for audio playback, among other patches.
\end{rSubsection}

\end{rSection}

%--------------------------------------------------
% INDEPENDENT PROJECTS
%--------------------------------------------------

\begin{rSection}{Independent Projects @ {\href{https://github.com/pranavrc/}{https://github.com/pranavrc}}}
\end{rSection}

\end{document}
