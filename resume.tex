%%%%%%%%%%%%%%%%%%%%%%%%%%%%%%%%%%%%%%%%%
% Medium Length Professional CV
% LaTeX Template
% Version 2.0 (8/5/13)
%
% This template has been downloaded from:
% http://www.LaTeXTemplates.com
%
% Original author:
% Trey Hunner (http://www.treyhunner.com/)
%
% Important note:
% This template requires the resume.cls file to be in the same directory as the
% .tex file. The resume.cls file provides the resume style used for structuring the
% document.
%
%%%%%%%%%%%%%%%%%%%%%%%%%%%%%%%%%%%%%%%%%

%----------------------------------------------------------------------------------------
%	PACKAGES AND OTHER DOCUMENT CONFIGURATIONS
%----------------------------------------------------------------------------------------

\documentclass{resume} % Use the custom resume.cls style

\usepackage[left=0.75in,top=0.5in,right=0.75in,bottom=0.75in]{geometry} % Document margins
\usepackage{hyperref}
\usepackage{xcolor}
\definecolor{darkred}{RGB}{139 0 0}

\hypersetup{
    colorlinks = true,
    urlcolor = darkred
}

\name{Pranav Ravichandran} % Your name
\address{\textit{Gainesville, FL} \\ \textit{{\href{mailto:me@onloop.net}{me@onloop.net}}} \\ \textit{{\href{http://onloop.net}{http://onloop.net}}} \\ \textit{{\href{http://github.com/pranavrc}{http://github.com/pranavrc}}}} % Your phone number and email

\begin{document}

%----------------------------------------------------------------------------------------
%	EDUCATION SECTION
%----------------------------------------------------------------------------------------

\begin{rSection}{Education}

{\bf Anna University} \hfill {2009-2013} \\
\textit{St. Joseph's College of Engineering, Electrical and Electronics Engineering Department}\\
Bachelor of Engineering in Electrical and Electronics Engineering. CGPA: 7.44/10

{\bf {\href{http://onloop.net/hairyplotter/}{Thesis, Final year}}} \hfill {2013} \\
\textit{Gestural interface with Electrooculogram-based eye gesture detection and classification.}

{\bf University of Florida} \hfill {Expected 2017 Grad} \\
\textit{College of Engineering, Computer and Information Science and Engineering Department} \\
Master of Science in Computer Science 

\end{rSection}

%----------------------------------------------------------------------------------------
%	TECHNICAL STRENGTHS SECTION
%----------------------------------------------------------------------------------------

\begin{rSection}{Technical Strengths}

\begin{tabular}{ @{} >{\bfseries}l @{\hspace{6ex}} l }
Languages & Python, Lisp, JavaScript, C \\
Tools \&Environment & Unix, Emacs, Bash, Git, Make/Cmake, Valgrind, Nginx, LaTeX \\
The Web & HTML, CSS, Django, Node.js, jQuery, Backbone.js \\
Databases & PostgreSQL, MongoDB
\end{tabular}

\end{rSection}

%----------------------------------------------------------------------------------------
%	WORK EXPERIENCE SECTION
%----------------------------------------------------------------------------------------

\begin{rSection}{Experience}

    \begin{rSubsection}{\href{https://www.google-melange.com/gsoc/project/details/google/gsoc2011/pranavrc/5757334940811264}{Google Summer of Code, Google}}{Apr - Sep 2011}{Student Developer}{Remote}
\item Integrated SMARTS, a game AI library, into KDE Gluon, giving it the capabilities of a Behavior Tree system.
\item Developed a KPart GUI to plug the SMARTS Behavior Tree editor into Gluon.
\end{rSubsection}

%------------------------------------------------

\begin{rSubsection}{\href{https://happyfox.com/}{HappyFox, Tenmiles}}{Jan 2014 - May 2015}{Core Developer}{Chennai, India}
\item Implemented features, enhancements and integrations for HappyFox, HappyFox Chat and related components.
\item Developed a decoupled REST-based task-server with Celery and RabbitMQ, to handle deferred services.
\item Created contextual search for knowledge base articles using Latent Semantic Analysis.
\end{rSubsection}

%------------------------------------------------

\end{rSection}

%-------------------------------------------------
% OPEN SOURCE CONTRIBUTIONS
%-------------------------------------------------

\begin{rSection}{Open Source Contributions}

\begin{rSubsection}{\href{https://bugzilla.mozilla.org/user\%5Fprofile?user\%5Fid=431664}{Mozilla}}{2012-2013}{}{}
Developed a code commenting and uncommenting feature for the Mozilla Firefox source editor, implemented W3C spec restrictions on WebIDL dictionary member types, and contributed other patches and fixes.
\end{rSubsection}

\begin{rSubsection}{\href{https://github.com/mdoege/PySynth/commits?author=pranavrc}{PySynth}}{July 2012}{}{}
Created an interactive Command Line Interpreter to parse Musical Notation and render tunes on the fly, implemented multi-library support for audio playback, among other patches.
\end{rSubsection}

\end{rSection}

%--------------------------------------------------
% INDEPENDENT PROJECTS
%--------------------------------------------------

\begin{rSection}{{\href{http://github.com/pranavrc/}{Independent Projects}}}

\item {\href{http://onloop.net/tenor/}{\bf Tenor}} A project to construct music theory and generate accessible music with Clojure and Overtone.
\item {\href{http://onloop.net/transit/}{\bf Transit.js}} An extensive JavaScript library to create purely client-side, interactive transit maps.
\item {\href{http://github.com/pranavrc/scrip/}{\bf Scrip}} A tiny domain-specific language described using Parsing Expression Grammar (PEG) to compute debts.
\item {\href{http://u.onloop.net/}{\bf Suburl}} A substitution-based URL sanitizer that allows shortening individual parts of the URL separately.
\item {\href{http://this.onloop.net/}{\bf This.onloop}} Music search and album discovery tool using the Spotify Music API.
\item {\href{http://redd.it/1i7lh3}{\bf Howstat}} Resident player-statistics lookup bot for Reddit's Cricket Community, /r/Cricket.

\end{rSection}


\iffalse
%-------------------------------------------------
% RELEVANT COURSEWORK
%-------------------------------------------------

\begin{rSection}{Relevant pre-requisite coursework, syllabi (where required) and grades}

\begin{rSubsection}{Fundamentals of Computing and Programming}{7/10}{}{}
\end{rSubsection}
\begin{rSubsection}{Computer Practice Laboratory - I}{10/10}{Office suites, C Programming, Pseudocode and Flowcharts}{}
\end{rSubsection}
\begin{rSubsection}{Computer Practice Laboratory - II}{10/10}{Unix, Unix Shell and Shell programming, C on Unix}{}
\end{rSubsection}
\begin{rSubsection}{Data Structures and Algorithms}{7/10}{}{}
\end{rSubsection}
\begin{rSubsection}{Data Structures and Algorithms Laboratory}{10/10}{}{}
\end{rSubsection}
\begin{rSubsection}{Object Oriented Programming}{5/10}{}{}
\end{rSubsection}
\begin{rSubsection}{Object Oriented Programming Laboratory}{10/10}{}{}
\end{rSubsection}
\begin{rSubsection}{Computer Networks}{7/10}{}{}
\end{rSubsection}
\begin{rSubsection}{Operating Systems}{7/10}{}{}
\end{rSubsection}
\begin{rSubsection}{Mathematics - I}{7/10}{Matrices, Three dimensional analytical geometry, Differential Calculus, Functions of several variables, Multiple integrals}{}{}
\end{rSubsection}
\begin{rSubsection}{Mathematics - II}{8/10}{Ordinary differential equations, Vector Calculus, Analytic functions, Complex integration, Laplace transform}{}{}
\end{rSubsection}
\begin{rSubsection}{Transforms and Partial Differential Equations}{8/10}{}{}
\end{rSubsection}
\begin{rSubsection}{Numerical Methods}{8/10}{}{}
\end{rSubsection}
\begin{rSubsection}{Microprocessors and Microcontroller}{5/10}{}{}
\end{rSubsection}
\begin{rSubsection}{Microprocessors and Microcontroller Laboratory}{7/10}{}{}
\end{rSubsection}

\end{rSection}
\fi


%----------------------------------------------------------------------------------------
%	EXAMPLE SECTION
%----------------------------------------------------------------------------------------

%\begin{rSection}{Section Name}

%Section content\ldots

%\end{rSection}

%----------------------------------------------------------------------------------------

\end{document}
